\section{gfal\_\-testget.c}
This exemple show how to convert a surl to a turl ex : use case with SRM plugin : ./gfal\_\-testget srm://grid05.lal.in2p3.fr:8446/dpm/lal.in2p3.fr/home/dteam/test\_\-darkvador



\begin{DocInclude}\begin{verbatim}1 
6 #include <stdio.h>
7 #include <stdlib.h>
8 #include <unistd.h>
9 #include <string.h>
10 #include <sys/types.h>
11 #include <errno.h>
12 #include <sys/param.h>
13 
14 #define DEFPOLLINT 10
15 #define TURL_MAX_SIZE 1024
16 
17 
18 #ifndef _GFAL_1_X
19 
20 #include <gfal_api.h>
21 
22  // NORMAL NEW WAY TO USE IN GFAL 2.0
23 
24 int main(int argc,char **argv)
25 {
26         int errsz = 500;
27         char err[errsz];
28 
29         char turl_buff[TURL_MAX_SIZE];
30 
31         if (argc < 2){
32                 fprintf (stderr, "usage: %s SURLs\n", argv[0]);
33                 exit (1);
34         }
35         gfal_set_verbose (GFAL_VERBOSE_VERBOSE);
36         
37         ssize_t res = gfal_getxattr(argv[1], "srm.turl", turl_buff,TURL_MAX_SIZE );
38         
39         if (res > 0)
40                 printf("SURL %s Ready - TURL: %s\n", argv[1], turl_buff);
41         else {
42                 printf("SURL %s Failed:\n%s\n", argv[1]);
43                 gfal_posix_check_error();
44         }
45                 
46         return((res>0)?0:-1);
47 }
48 
49 #endif
50 
51 
52 /*  This is the OLD WAY GFAL 1.0 to do IT, DONT USE IT WITH GFAL2.0, IT IS FOR A  EXAMPLE OF MIGRATION 1.X to 2.0
53 #ifdef _GFAL_1_X 
54 #define gfal_handle_free(x) gfal_internal_free(x)
55 
56 
57 #include <gfal_api.h>
58 
59 main(argc, argv)
60 int argc;
61 char **argv;
62 {
63         gfal_request req = NULL;
64         gfal_internal gobj = NULL;
65         gfal_filestatus *filestatuses = NULL;
66         int sav_errno = 0, n = 0, i = 0, nberrors = 0;
67         static char *protos[] = {"rfio", "dcap", "gsiftp"};
68 
69         if (argc < 2) {
70                 fprintf (stderr, "usage: %s SURLs\n", argv[0]);
71                 exit (1);
72         }
73 
74         gfal_set_verbose (0);
75 
76         if ((req = gfal_request_new ()) == NULL)
77                 exit (1);
78         req->nbfiles = argc - 1;
79         req->surls = argv + 1;
80         req->protocols = protos;
81 
82         if (gfal_init (req, &gobj, NULL, 0) < 0) {
83                 sav_errno = errno;
84                 free (req);
85                 errno = sav_errno;
86                 perror (argv[0]);
87                 exit (1);
88         }
89         free (req);
90 
91         if (gfal_turlsfromsurls (gobj, NULL, 0) < 0) {
92                 sav_errno = errno;
93                 gfal_internal_free (gobj);
94                 errno = sav_errno;
95                 perror (argv[0]);
96                 exit (1);
97         }
98 
99         if ((n = gfal_get_results (gobj, &filestatuses)) < 0) {
100                 sav_errno = errno;
101                 gfal_internal_free (gobj);
102                 errno = sav_errno;
103                 perror (argv[0]);
104                 exit (1);
105         }
106 
107         if (filestatuses == NULL) {
108                 fprintf (stderr, "%s: Internal error (memory corruption?)\n", argv[0]);
109                 exit (1);
110         }
111 
112         for (i = 0; i < n; ++i) {
113                 if (filestatuses[i].status == 0)
114                         printf("SURL %s Ready - TURL: %s\n", filestatuses[i].surl, filestatuses[i].turl);
115                 else {
116                         ++nberrors;
117                         if (filestatuses[i].explanation)
118                                 printf("SURL %s Failed:\n%s\n", filestatuses[i].surl, filestatuses[i].explanation);
119                         else
120                                 printf("SURL %s Failed:\n%s\n", filestatuses[i].surl, strerror (filestatuses[i].status));
121                 }
122         }
123 
124         gfal_internal_free (gobj);
125         exit (nberrors > 0 ? 1 : 0);
126 }
127 #endif
128 */
\end{verbatim}
\end{DocInclude}
 